\chapter{Prestazioni dei sistemi di controllo}
	
	In generale la funzione di trasferimento $\G_f$ di un sistema retro-azionato dipende sia dalla funzione di trasferimento $R(s)$ del \textbf{regolatore}, sia dalla funzione di trasferimento $G(s)$ del sistema controllato.
	
	In generale un sistema presenta l'ingresso di riferimento $\overline y$, i disturbi $d$ e \textbf{boh} $n$ che, tramite una notazione matriciale della funzione di trasferimento $\G_f(s)$ permette di calcolare l'uscita $y$, l'errore $e$ e l'uscita $u$ (in ingresso al sistema).
	
	Osservando lo schema del sistema da analizzare è possibile calcolare le singole funzioni di trasferimento che legano i vari ingressi con le varie uscite, osservando che
	\[ \overline y \rightarrow y \quad: \quad  F(s) = \frac L {1+L} \qquad \overline y \rightarrow u \quad: \quad Q(s) = \frac R {1+L} \qquad S(s) = \frac 1 {1+L} \]
	
	\textbf{RIVEDERE}
	
	\[ \begin{pmatrix}
		 y\\ e \\ u
	\end{pmatrix} = \begin{bmatrix}
	 	F(s)  & S(s) & -F(s) \\ S(s) & - S(s) & F(s) \\ Q(s) & - Q(s) & - Q(s) 
	\end{bmatrix}  \begin{pmatrix}
		\overline y \\ d \\ n
	\end{pmatrix}\]
	
	La funzione $S(s)$ viene denominata \de{funzione di sensitività} (\textbf{PERCHE?!}), $F(s)$ è la \de{funzione di sensitività complementare} (complementare in quanto $S + F = 1$) e $Q(s)$ è la \de{funzione di sensibilità del controllo} (in quanto compare solamente per calcolare la variabile di controllo $u$).
	
	Si osserva che in ogni funzione di trasferimento $F,S,Q$ il denominatore è comune e pari a $1 + L$, con $L$ funzione d'anello (che giustamente è indipendente dall'ingresso/uscita considerato per il sistema). Inoltre le funzioni $F(s)$ e $S(s)$ dipendono esplicitamente solo da $L(s)$. Questo significa che l'analisi delle prestazioni è diversa rispetto a quanto descritto nel capitolo precedente; in particolare scelto il regolare $R$ è possibile calcolare le caratteristiche di $L$ e dunque delle funzioni $F$ ed $S$. 
	
	\paragraph{Compromesso sulla scelta del regolatore} RISCRIVERE
	
\section{Analisi delle prestazioni}

	\subsection{Funzione di sensitività complementare}
		In generale la funzione di sensitività complementare $F(s)$ permette di relazionare il valore desiderato $\overline y$ con l'uscita $y$ del sistema, e per migliorare le prestazioni si vuol far si che $F(s)$ si comportasse come un filtro passa basso di guadagno unitario.
		
		A questo punto per collegare le caratteristiche di $F(s)$ con il guadagno d'anello è necessario esplicitare l'espressione di quest'ultimo:
		\[ F(s) = \frac L {1+L} = \frac{N_L/D_L}{1+ N_L/D_L} = \frac{N_L}{D_L + N_L} \]
		Da questa espressione è possibile verificare come gli zeri di $F(s)$ coincidano con quelli di $L(s)$; volendo esprimere in genere $F(s)$ con un'espressione del tipo
		\[ F(s) \backsim \frac {\mu_f} {s^g} \frac{\textrm{zeri dominanti}}{\textrm{poli dominanti}}\]
		
		Per ricavare il valore $\mu_f / s^g$ è necessario studiare il limite per $s$ che tende a zero della funzione $F(s)$:
		\[ \frac{\mu_f}{s^g} =  \lim_{s\rightarrow 0} F(s) = \lim_{s\rightarrow 0} \frac{L(s)}{1+L(s)} = \lim_{s\rightarrow 0} \frac{\mu_l / s^{g_l}}{ 1 + \mu_l / s^{g_l}} \]	
		A questo punto è possibile osservare come le proprietà a bassa frequenza della sensitività complementare $F(s)$ (ossia il rapporto $\mu/s^g$) dipende direttamente dalle proprietà a bassa frequenza del guadagno d'anello $L(s)$; in particolare effettuando delle semplificazioni è possibile calcolare il valore dei parametri della sensitività complementare:
		\[ \lim_{s\rightarrow 0} \frac{\mu_l / s^{g_l}}{ 1 + \mu_l / s^{g_l}} \quad \rightarrow \quad \begin{cases}
			\mu_f = \frac{\mu_l}{1 + \mu_l} \qquad & \textrm{se } g_l = 0 \\
			\mu_f = 1 \qquad & \textrm{se } g_l \geq 1 \\ 
			\mu_f = \mu_l \textrm{ e } g_f = g_l  \qquad & \textrm{se } g_l < 0 \\ 
		\end{cases}  \]
	
		Si osserva dunque che sono presenti uno o più poli nella funzione d'anello nell'origine ($g_l \leq 1$), allora ad esso è associato un termine di tipo integrativo $1/s$ che risulterà essere utile in quanto nei sistemi di controllo PID garantiranno l'unitarietà del guadagno della funzione d'anello.
		
		\vspace{3mm}
		Calcolare esplicitamente i poli della funzione di sensitività complementare $F(s)$ risulta essere piuttosto difficile  e per questo essi vengono dedotti analizzando il diagramma del modulo di Bode: in particolare se la pendenza dello stesso cala allora si ha un polo, mentre se la pendenza sale si ha uno zero.
		Considerando la relazione esplicita della sensitività, si dimostra che il modulo della stessa dipende direttamente dal guadagno della funzione d'anello.
		\[ |F(j\omega)| = \frac{|L(j\omega)|}{|1+L(j\omega)|} = f\big(|L(j\omega)|\big) \]
		Nell'ipotesi in cui il guadagno d'anello $|L|$ è molto maggiore del valore unitario (e dunque in decibel $|L|_{dB} \gg 0 $), allora è possibile considerare che
		\[ |F| \approx \frac{|L|}{|L|} = 1 \qquad \textrm{se } |L| \gg 1 \leftrightarrow |L|_{dB} \gg 0 \]
		Al contrario è possibile effettuare la semplificazione inversa nel caso in cui il guadagno sia molto inferiore del valore unitario e dunque
		\[ |F| \approx \frac{|L|}{1} = |L| \qquad \textrm{se } |L| \ll 1 \leftrightarrow |L|_{dB} \ll 0 \]
		
		
\section{Sintesi di sistemi di controllo}