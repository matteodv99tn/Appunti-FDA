\chapter{Prestazioni dei sistemi di controllo}
	
	In generale la funzione di trasferimento $\G_f$ di un sistema retro-azionato dipende sia dalla funzione di trasferimento $R(s)$ del \textbf{regolatore}, sia dalla funzione di trasferimento $G(s)$ del sistema controllato.
	
	In generale un sistema presenta l'ingresso di riferimento $\overline y$, i disturbi $d$ e \textbf{boh} $n$ che, tramite una notazione matriciale della funzione di trasferimento $\G_f(s)$ permette di calcolare l'uscita $y$, l'errore $e$ e l'uscita $u$ (in ingresso al sistema).
	
	Osservando lo schema del sistema da analizzare è possibile calcolare le singole funzioni di trasferimento che legano i vari ingressi con le varie uscite, osservando che
	\[ \overline y \rightarrow y \quad: \quad  F(s) = \frac L {1+L} \qquad \overline y \rightarrow u \quad: \quad Q(s) = \frac R {1+L} \qquad S(s) = \frac 1 {1+L} \]
	
	\textbf{RIVEDERE}
	
	\[ \begin{pmatrix}
		 y\\ e \\ u
	\end{pmatrix} = \begin{bmatrix}
	 	F(s)  & S(s) & -F(s) \\ S(s) & - S(s) & F(s) \\ Q(s) & - Q(s) & - Q(s) 
	\end{bmatrix}  \begin{pmatrix}
		\overline y \\ d \\ n
	\end{pmatrix}\]
	
	La funzione $S(s)$ viene denominata \de{funzione di sensitività} (\textbf{PERCHE?!}), $F(s)$ è la \de{funzione di sensitività complementare} (complementare in quanto $S + F = 1$) e $Q(s)$ è la \de{funzione di sensibilità del controllo} (in quanto compare solamente per calcolare la variabile di controllo $u$).
	
	Si osserva che in ogni funzione di trasferimento $F,S,Q$ il denominatore è comune e pari a $1 + L$, con $L$ funzione d'anello (che giustamente è indipendente dall'ingresso/uscita considerato per il sistema). Inoltre le funzioni $F(s)$ e $S(s)$ dipendono esplicitamente solo da $L(s)$. Questo significa che l'analisi delle prestazioni è diversa rispetto a quanto descritto nel capitolo precedente; in particolare scelto il regolare $R$ è possibile calcolare le caratteristiche di $L$ e dunque delle funzioni $F$ ed $S$. 
	
	\paragraph{Compromesso sulla scelta del regolatore} RISCRIVERE
	
\section{Analisi delle prestazioni}

	\subsection{Funzione di sensitività complementare}
		In generale la funzione di sensitività complementare $F(s)$ permette di relazionare il valore desiderato $\overline y$ con l'uscita $y$ del sistema, e per migliorare le prestazioni si vuol far si che $F(s)$ si comportasse come un filtro passa basso di guadagno unitario.
		
		A questo punto per collegare le caratteristiche di $F(s)$ con il guadagno d'anello è necessario esplicitare l'espressione di quest'ultimo:
		\[ F(s) = \frac L {1+L} = \frac{N_L/D_L}{1+ N_L/D_L} = \frac{N_L}{D_L + N_L} \]
		Da questa espressione è possibile verificare come gli zeri di $F(s)$ coincidano con quelli di $L(s)$; volendo esprimere in genere $F(s)$ con un'espressione del tipo
		\[ F(s) \backsim \frac {\mu_f} {s^g} \frac{\textrm{zeri dominanti}}{\textrm{poli dominanti}}\]
		
		Per ricavare il valore $\mu_f / s^g$ è necessario studiare il limite per $s$ che tende a zero della funzione $F(s)$:
		\[ \frac{\mu_f}{s^g} =  \lim_{s\rightarrow 0} F(s) = \lim_{s\rightarrow 0} \frac{L(s)}{1+L(s)} = \lim_{s\rightarrow 0} \frac{\mu_l / s^{g_l}}{ 1 + \mu_l / s^{g_l}} \]	
		A questo punto è possibile osservare come le proprietà a bassa frequenza della sensitività complementare $F(s)$ (ossia il rapporto $\mu/s^g$) dipende direttamente dalle proprietà a bassa frequenza del guadagno d'anello $L(s)$; in particolare effettuando delle semplificazioni è possibile calcolare il valore dei parametri della sensitività complementare:
		\[ \lim_{s\rightarrow 0} \frac{\mu_l / s^{g_l}}{ 1 + \mu_l / s^{g_l}} \quad \rightarrow \quad \begin{cases}
			\mu_f = \frac{\mu_l}{1 + \mu_l} \qquad & \textrm{se } g_l = 0 \\
			\mu_f = 1 \qquad & \textrm{se } g_l \geq 1 \\ 
			\mu_f = \mu_l \textrm{ e } g_f = g_l  \qquad & \textrm{se } g_l < 0 \\ 
		\end{cases}  \]
	
		Si osserva dunque che sono presenti uno o più poli nella funzione d'anello nell'origine ($g_l \leq 1$), allora ad esso è associato un termine di tipo integrativo $1/s$ che risulterà essere utile in quanto nei sistemi di controllo PID garantiranno l'unitarietà del guadagno della funzione d'anello.
		
		\vspace{3mm}
		Calcolare esplicitamente i poli della funzione di sensitività complementare $F(s)$ risulta essere piuttosto difficile  e per questo essi vengono dedotti analizzando il diagramma del modulo di Bode: in particolare se la pendenza dello stesso cala allora si ha un polo, mentre se la pendenza sale si ha uno zero.
		Considerando la relazione esplicita della sensitività, si dimostra che il modulo della stessa dipende direttamente dal guadagno della funzione d'anello.
		\[ |F(j\omega)| = \frac{|L(j\omega)|}{|1+L(j\omega)|} = f\big(|L(j\omega)|\big) \]
		Nell'ipotesi in cui il guadagno d'anello $|L|$ è molto maggiore del valore unitario (e dunque in decibel $|L|_{dB} \gg 0 $), allora è possibile considerare che
		\[ |F| \approx \frac{|L|}{|L|} = 1 \qquad \textrm{se } |L| \gg 1 \leftrightarrow |L|_{dB} \gg 0 \]
		Al contrario è possibile effettuare la semplificazione inversa nel caso in cui il guadagno sia molto inferiore del valore unitario e dunque
		\[ |F| \approx \frac{|L|}{1} = |L| \qquad \textrm{se } |L| \ll 1 \leftrightarrow |L|_{dB} \ll 0 \]
		
		A questo punto per progettare un buon regolatore si richiede che la funzione d'anello sia di tipo positivo ($g>0$), oppure se lo stesso è nullo si vuole avere un guadagno d'anello molto alto.
		
		\paragraph{Ricerca dei poli dominanti} Solitamente i poli dominanti sono quelli \textit{lenti}, ossia con pulsazione $\omega$ piccola; tuttavia questa \textit{regola} non è generale. In particolare se lo smorzamento $\xi$ di un polo a pulsazione veloce è particolarmente piccolo, allora è possibile che tale polo sia dominante (in quanto si ha un picco di risonanza particolarmente pronunciato).
		
		In generale nel progetto di sistemi di controllo  non si vuole mai avere un picco di risonanza, e in generale si vuole che il sistema si comporti come un filtro passa basso. Questo comporta una semplificazione sui diagrammi di Bode e la pulsazione del filtro passa basso associato alla funzione di sensitività complementare $F$ coincide con la pulsazione critica $\omega_c$ del guadagno d'anello.
		
		A questo punto è necessario capire sono i poli sono reali oppure complessi coniugati (e determinarne dunque lo smorzamento). Per rispondere alla domanda si parte calcolando l'eventuale smorzamento dei poli (supponendo che esista) considerando che il picco del modulo in corrispondenza della pulsazione critica vale
		\[ |F(j\omega_c)| = \frac{1}{2|\xi|} \qquad \Rightarrow \quad \xi = \sin\big(\phi_m/2\big) \approx \frac{\phi_m[^\circ]}{100} \]
		Per effettuare questa operazione è possibile verificare i seguenti passaggi:
		\begin{align*}
			|F(j\omega_c)| & = \frac{|L(j\omega_c)|}{|1+L(j\omega_c)|} \xrightarrow[L(j\omega_c) = e^{j\phi_c}]{|L(j\omega_c)| =1 } \frac{1}{|1 + e^{j\phi_c}|} = \frac 1  {\sqrt{\big(1 + \cos^2\phi_c\big) + \sin^2\phi_c}} \\
			& = \frac 1 {\sqrt{2\big(1+ \cos\phi_c\big)}} = \frac{1}{\sqrt{2 \big(1-\cos \phi_m\big)}} = \frac 1 {2\sin\big(\phi_m/2\big)}
		\end{align*}
		Dal risultato che otteniamo nel calcolo dello smorzamento $\xi$ permette di stabilire dunque se i poli sono reali o complessi coniugati; in particolare se $\xi\leq \frac {\sqrt 2 } 2$ allora i poli possono essere considerati come complessi coniugati, mentre se $\xi > \frac {\sqrt 2} 2$ allora i poli sono considerati come reali; ciò significa che per $\phi_m < 75^\circ$ allora i poli sono complessi coniugati (per $\phi_m > 75^\circ$ i poli sono considerati reali coincidenti). \\
		A questo punto noto che il tempo di assestamento vale $t_a = 5 / |Re|$, allora per $\phi_m > 75^\circ$ allora si ha un polo reale con tempo di assestamento $t_a = 5 /\omega_c$, mentre se $\phi_m < 75^\circ$ (2 poli complessi coniugati) allora
		\[ t_a = \frac{5}{\omega_c \xi} = \frac{5}{\omega_c \frac{\phi_m}{100}} \]
		
	\subsection{Funzione della sensitività}
		
		Si analizza ora la sensitività $S(s)$ che permette di correlare il riferimento $\overline y$ con l'errore $e$ di inseguimento. Tale funzione può essere riscritta come
		\[ S(s) = \frac 1 {1+L}  = \frac{D_L}{D_L+N_L} \]
		Si osserva così che gli zeri di $S$ coincidono con i poli della funzione d'anello $L$, mentre i poli di $S$ coincidono con quelli delal sensitività complementare.
		
		A questo punto di determina guadagno e tipo analizzando il limite per $s\rightarrow 0$ (come nel caso precedente) arrivando ai risultati
		\[ \lim_{s\rightarrow 0} \frac{1}{ 1 + \mu_l / s^{g_l}} \quad \rightarrow \quad \mu_s = \begin{cases}
			\frac{1}{1 + \mu_l} \textrm{ e } g_s = 0 \qquad & \textrm{se } g_l = 0 \\
			 1/\mu_l \textrm{ e } g_s = - g_l \qquad & \textrm{se } g_l \geq 1 \\ 
			 1 \textrm{ e } g_s = 0  \qquad & \textrm{se } g_l < 0 \\ 
		\end{cases}  \]
		Questi risultati sono compatibili rispetto a quanto visto nell'analisi della sensitività complementare: infatti se $g_l = 0$ l'uscita $y$ tende asintoticamente al valore di riferimento $\overline y$ e conseguentemente l'errore $e$ deve convergere a zero (considerazioni analoghe possono essere svolte per tutte le casistiche).
		
		\paragraph{Principio del modello interno} Considerando di calcolare l'errore asintotico $e_\infty$ per un ingresso di riferimento $\overline y(s) = A / s^r$, allora per il teorema del valore finale si osserva che
		\[ e_\infty = \lim_{s\rightarrow 0} sE(s) = \lim_{s\rightarrow 0} s S(s) \frac A{s^r}  = \lim_{s\rightarrow 0} S(s) \frac A{s^{r-1}} \]
		Nel caso in cui $g_l \geq 1$ allora il limite si può riscrivere come
		\[ \lim_{s\rightarrow 0} \frac 1  {\mu_l} s^{g_l} \frac A {s^{r-1}} = \lim_{s\rightarrow 0} \frac A {\mu_l} s^{g_l - r + 1} \]
		
		Ponendo $r=1$ (associato ad un riferimento a scalino) allora per avere un errore asintotico nullo è necessario imporre che $g_l \geq 1$; se invece si sceglie $r=2$ (associato ad un riferimento a rampa) per avere errore asintotico nullo si deve avere $g_l \geq 2$. In generale fissato il valore di $r$, allora per avere errore asintotico nullo si deve avere $g_l \geq r$.
		
		In generale il \textbf{principio del modello interno} afferma che \textit{per ottenere un errore asintotico nullo, a fronte di un riferimento $\overline y(s)$, occorre che in $L(s)$ sia presente il modello di $\overline y(s)$.}
		
		Studiando la posizione di poli e zeri della funzione di sensitività $S(s)$ è possibile analizzare la risposta in frequenza del sistema; in particolare studiando il modulo  si osserva che lo stesso è funzione del modulo della funzione d'anello:
		\begin{align*}
			|S(j\omega)| & = \frac 1 {|1 + L(j\omega)|} = f\big(|L(j\omega)|\big) \\
			& \approx \begin{cases}
				|S| = \frac 1 {|L|} \ \leftrightarrow \ |S|_{dB} = - |L|_{dB} \qquad & \textrm{se } |L|\gg 1, \omega \ll \omega_c \\
				|S| = 1 \ \leftrightarrow \ |S|_{dB} = 0dB \qquad & \textrm{se } |L|\ll 1, \omega \gg \omega_c \\
			\end{cases}
		\end{align*}
		
		Si osserva dunque che il comportamento della funzione di sensitività si comporta come un filtro passa alto con frequenza di taglio $\omega_c$ coincidente con quella della funzione di sensitività complementare $F(s)$ (che si comporta tuttavia come un filtro passa basso). Questo significa che i poli della funzione di sensitività $S(s)$ coincidono con quelli della funzione di sensitività complementare $F(s)$. Questo può anche essere intuitivamente dimostrato considerando che il denominatore di $S(s)$ e $F(s)$ sono uguali, e dunque i poli non possono che non essere coincidenti.
		
	\subsection{Funzione della sensitività del controllo}
		La \de{sensitività di controllo} $Q(s)$ permette di correlare il riferimento $\overline y$, disturbi e rumori $d,n$ con l'ingresso al sistema controllato $u$; in particolare si era dimostrato valere che tale funzione è calcolata come $Q = \frac R {1+L} = f(R,L)$, ossia dipende sia dal guadagno d'anello ma anche dalla funzione di trasferimento dovuta al regolatore $R$. Questo fatto rende più \textit{difficoltoso} approcciare il progetto del controllore per come fin'ora introdotto, agendo di fatto solo su $L(s)$.
		
		D'altra parte è possibile anche osservare che le caratteristiche di $Q(s)$ (che determina di fatto l'andamento di $u$) sono \textit{secondarie} rispetto ai risultati delle funzioni $F,S$ che determinano l'andamento dell'uscita $y$ e dell'errore $e$. Nella pratica dunque si progetta il regolare in modo da ricavare dei risultati desiderabili su $y,e$, mentre il calcolo di $Q(s)$ si fa \textit{a posteriori} solamente per dimensionare l'\textbf{azione di controllo}, ossia gli attuatori.
		
		\paragraph{Nota} E' in realtà possibile riscrivere la formulazione di $G(s)$ in maniera più \textit{agevola}; in particolare si dimostra che
		\begin{align*}
			Q(s) & = \frac R {1+L} \frac G G = \frac L {1+L} G^{-1} = F(s) G^{-1}(s) \\
			\Rightarrow \qquad F(s)& = G(s) Q(s) 
		\end{align*}
		Da questa relazione è possibile calcolare $|F|_{dB} = |G|_{dB} + |Q|_{dB}$; per \textit{velocizzare} un sistema dinamico è necessario \textit{spostare} verso $\omega$ più alte il polo della funzione $F(s)$; essendo il guadagno $G(s)$ del sistema da controllare prestabilito, è necessario compensare la differenza con la presenza di una sensitività di controllo $Q(s)$ tale da mantenere le relazioni sul modulo appena viste.
		
		In generale quando $|F|\backsim|G|$ l'azione del sistema di controllo $Q$ è piccola, mentre diventa valore diventa importante nei punti in cui le funzioni di trasferimento di $F$ e $G$ diventano diverse: maggiore è la modifica di $G$, maggiore è l'azione di controllo necessario.
		
		\begin{concetto}
			Per cambiare la dinamica di un sistema non basta \textit{progettare bene} un regolatore, ma occorre anche dell'\textit{energia}, ossia l'azione di controllo, tanto maggiore quanto sono le modifiche al sistema originario.
		\end{concetto}
		
		
\section{Sintesi di sistemi di controllo}
	La \de{sintesi} è il processo di progettazione del regolatore in modo che il sistema controllato si comporti come richiesto; in particolare nel dominio di Laplace un processo di sintesi di un sistema può essere caratterizzato dai seguenti passaggi:
	\begin{itemize}
		\item si scelgono le caratteristiche desiderate del sistema, quali le risposte dell'uscita $y$, dell'errore $e$...
		\item si traducono le caratteristiche desiderate in proprietà desiderate della funzione d'anello $L(s)$;
		\item si disegna/progetta/calcola una funzione d'anello $L(s)$ compatibile con le caratteristiche appena determinate;
		\item nota la funzione di trasferimento $G(s)$ del sistema da controllare, si calcola il regolare tramite inversione del guadagno d'anello:
		\[ L(s) = R(s) G(s) \qquad \rightarrow \quad R (s) = \frac{L(s)}{G(s)}\]
	\end{itemize}	
	Questo processo prende il nome di \de{sintesi di $R(s)$ tramite il \textit{loop shaping}}.

	\subsection{Stabilità del sistema di controllo}
		Il primo passaggio nella sintesi del sistema di controllo è assicurarsi che lo stesso, alla fine del processo, risulti asintoticamente stabile:
		\begin{itemize}
			\item se il sistema da controllare era già in origine asintoticamente stabile (per cui il numero di poli a parte reale positiva $P = 0 $ sia nulla) è sufficiente analizzare la stabilità tramite il criterio di Bode. Si deve dunque garantire, per la funzione d'anello, che sia il guadagno $\mu_L"$ che margine di fase $\phi_m$ siano positivi non nulli e che l'attraversamento del diagramma del modulo di Bode attraversi l'intercetta orizzontale una sola volta \textit{dall'alto verso il basso}.
			
			\item se il sistema da controllare è invece instabile in partenza il progetto del regolare diventa più complesso e si può effettuare utilizzando delle tecniche di progettazione diverse (che si vede alla magistrale) oppure utilizzando il metodo degli \textit{anelli in cascata} (ossia si crea un anello interno il cui scopo è quello di stabilizzare il sistema, mentre l'anello esterno su cui si innesta il primo viene progettato per ottenere i risultati desiderati).
			
		\end{itemize}
	
	\subsection{Specifiche del sistema}
		La scelta delle \textbf{specifiche del sistema} possono essere suddivise in quelle \textbf{asintotiche} (o \textit{statiche}) e \textbf{dinamiche}.
		
		\paragraph{Specifiche asintotiche} Le specifiche asintotiche studiano il comportamento del sistema controllato a livello asintotico, ossia per $t\rightarrow \infty$, per ingressi \textit{canonici} (come lo scalino o la rampa, ossia segnali per cui vale il teorema del valore finale) e per ingressi sinusoidali (ossia per ingressi per cui è necessario considerare il teorema della risposta in frequenza).
		
		\vspace{3mm}
		
		Esempi di specifiche asintotiche per ingressi canonici possono essere che l'errore siano compresi entro degli intervalli specifici ($|e_\infty| <h$, $y_\infty > h$). A questo punto è possibile eseguire i seguenti passaggi:
		\begin{itemize}
			\item si calcola la funzione di trasferimento (chiamata, per esempio,$W(s)$) che lega l'ingresso all'uscita, ossia che permette di calcolare le funzioni $F(s), S(s), \dots$
			\item applico il teorema del valore finale; considerando l'esempio di voler stabilire l'errore asintotico è sufficiente calcolare
			\[ |e_\infty| = \lim_{s\rightarrow 0} s W(s) \overline y(s) \]
			\item a questo punto si ottiene un'espressione funzione dei parametri $g_L,\mu_L$ del guadagno d'anello: a questo punti si impone la specifica e si trovano i valori di $g_L$ e $\mu_L$.
		\end{itemize}
		
		\vspace{3mm}
		
		Considerando invece segnali in ingressi sinusoidali il processo è analogo, tuttavia non si applica più il teorema del valore finale ma quello della risposta in frequenza.
		
		\textbf{MINUTO 26}
				
		

































