\chapter{Sistemi dinamici interconnessi}
\section{Schemi a blocchi}
	\marginnote{10 maggio}
	La discussione sui \textbf{sistemi dinamici interconnessi} è la base per capire come funzionano i sistemi in retroazione e si basa sullo studio del comportamento derivante dal collegamento di 2 o più sistemi dinamici.
	
	Per capire come i sistemi sono interconnessi si ricorre al formalismo grafico degli \textbf{schemi a blocchi}; questo tipo di rappresentazione permette di analizzare anche connessioni molto complesse in maniera sistematica e relativamente \textit{semplice}.
	
	Gli elementi costitutivi della rappresentazione sono:
	\begin{itemize}
		\item il \textbf{blocco} che rappresenta il sistema dinamico; esso è caratterizzato dai suoi ingressi (a sinistra), dalle uscite (a destra) e presenta all'interno la sua funzione di trasferimento $\G$ che permette di scrivere l'\textbf{\textit{equazione costitutiva}} $Y = \G U$ del sistema;
		\item il \textbf{nodo sommatore}, rappresentato da un pallino, nel quale entrano le \textit{frecce} (con eventuale segno negativo se i valori devono essere invertiti) che devono essere sommate rispetto all'unica frecca uscente dal nodo;
		\item la \textbf{diramazione}, ossia una \textit{linea} la cui estremità è una pallino dal quale partono altre linee di valore uguale alla diramazione in ingresso.
	\end{itemize}
	
	\subsection{Connessioni elementari}
		Tra tutte le connessioni che si possono realizzare, è possibile individuarne 3 di fondamentali:
		\begin{itemize}
			\item esiste la \textbf{connessione in serie}, dove \textit{due scatole vengono poste una in fila all'altra}, ossia l'uscita della prima coincide con l'ingresso della seconda (e il processo si può iterare). La loro connessione risulta in un'altro sistema lineare la cui funzione di trasferimento vale
			\[ \G_\textrm{serie} = \prod_i \G_i  \]
			Tale espressione, riferita ad un sistema di due blocchi, può essere dimostrata esplicitando le equazioni costitutive dei 2 blocchi e l'equazione della diramazione:
			\[ \begin{cases}
				Y_2 = \G_2 U_2 \\ Y_1 = \G_1 U_1 \\ U_2 = Y_1 
			\end{cases} \qquad Y = Y_2 = \G_\textrm{serie} U = \G_2\G_1 U_1\]
			
			\item la seconda connessione fondamentale è il \textbf{parallelo} dove uno stesso ingresso viene imposto a due (o più) blocchi le cui uscite si sommano; si dimostra che la funzione di trasferimento equivalente $\G_\textrm{parallelo}$ realizzata dai blocchi in parallelo vale
			\[ \G_\textrm{parallelo} = \sum_i \G_i \]
			Considerando il caso di 2 blocchi in parallelo, allora è possibile scrivere l'equazione del nodo sommatore $Y = Y_1 + Y_2$, mentre note le due equazioni costitutive $Y_i = \G_i U_i$ e noto che $U_1 = U_2 = U$, allora
			\[ Y = \big(\G_1 + \G_2\big)U \]
			
			\item la connessione in \textbf{retroazione}, come in \textbf{figura}, può essere analizzata a partire dalle equazioni caratteristiche del problema:
			\[ \begin{cases}
				Y_1 = Y \\ U_2 = Y \\ Y_1 = \G_1 W \\ W = U \pm Y_2 \\ Y_2 = \G_2 U_2
			\end{cases} \qquad \Rightarrow \quad Y = \underbrace{\frac{\G_1}{1 \mp \G_1\G_2}}_{\G_\textrm{feedback}} U \]
			In generale la funzione di trasferimento di un qualsiasi sistema di trasferimento è dato dal rapporto della funzione di trasferimento della \textit{linea di mandata} e la somma tra 1 e la \textit{funzione d'anello}:
			\[ \G_\textrm{feedback} = \frac{\textrm{f.d.t. linea di andata}}{1 \mp \textrm{funzione d'anello}}\]
			
		\end{itemize}
	
	\subsection{Rappresentazione dei sistemi dinamici}
		In generale i sistemi dinamici rappresentano come ingressi sia l'ingresso \textit{proprio} $u$, ma anche un disturbo $d$: il sistema dinamico dunque può essere modellato come un parallelo di un blocco contenente la funzione di trasferimento modellante l'ingresso $\G_u$ e quello dovuto al disturbo $\G_d$ (questo vale per sistemi lineari tempo invarianti, non in generale per altri sistemi).
		
		Nei sistemi dinamici in retroazione è possibile anche inserire gli effetti, sul ramo in retroazione, di rumore come un blocco con funzione di trasferimento $\G_n = \G_\textrm{noise}$. In generale nello studio dei sistemi di controllo si trascurano in prima battuta gli effetti dovuti ai disturbi e ai rumori.
		
		
\section{Stabilità di sistemi interconnessi}
	
	\paragraph{Connessione in serie e in parallelo} Considerando che la funzione di trasferimento di un generico sistema lineare tempo invariante può essere espresso come un rapporto di polinomi $\G_i = N_i/D_i$, allora è possibile osservare che la \textbf{funzione di trasferimento} complessiva della \textbf{serie}  è una funzione polinomiale che ha come zeri l'unione degli zeri delle due funzioni iniziali, e come poli l'unione dei due poli iniziali:
	\[ \G_\textrm{serie} = \G_1 \G_2 = \frac{N_1}{D_1} \frac{N_2}{D_2} = \frac N D \]
	
	Considerando invece la \textbf{connessione in parallelo} non è possibile calcolare gli zeri come diretta conseguenza degli zeri dei blocchi di partenza, tuttavia i poli è possibile osservare che coincidono con l'unione dei poli iniziali:
	\[ \G_\textrm{parallelo} = \G_1 + \G_2 = \frac{N_1}{D_1} + \frac{N_2}{D_2} = \frac{N_1D_2+N_2D_1}{D_1D_2} = \frac N D  \]
	
	In realtà essendo importanti per la stabilità soprattutto la posizione dei poli, è possibile concludere che connessioni in serie/parallelo di due sistemi asintoticamente stabili risulta in un sistema che è anch'esso asintoticamente stabile. Ne consegue che se anche uno dei due sistemi è instabile, anche allora il sistema complessivo lo sarà: la connessione in serie/parallelo dunque non permette di stabilizzare un sistema dinamico inizialmente instabile, in ogni caso!
	
	\paragraph{Sistemi retro-azionati} Si dimostra dunque che l'unico collegamento \textit{utile} a stabilizzare un sistema instabile è il collegamento in retroazione; esplicitando infatti le funzioni di trasferimento $\G_1$ del ramo di mandata e $\G_2$ nel ramo in retroazione è possibile calcolare la funzione di trasferimento complessiva come
	\[ \G_\textrm{feedback} = \frac{\G_1}{1\mp \G_1\G_2} = \frac{N_1/D_1}{1 \mp \dfrac{N_1}{D_1}\dfrac{N_2}{D_2}} = \frac{N_1D_2}{D_1D_2 \mp N_1N_2} = \frac N D \]
	Osservando il denominatore $D$ della funzione di trasferimento risultante è possibile notare che i nuovi poli ottenuti non coincidono, in maniera generale, con le radici del polinomio di partenza; dunque non è detto che il collegamento di due sistemi asintoticamente stabili non è detto che sia anch'esso asintoticamente stabile, tuttavia un sistema instabile può essere stabilizzato tramite un'opportuna retroazione con un altro sistema dinamico.
	
	\subsection*{Criteri di stabilità per sistemi retro-azionati}
		La stabilità di un sistema retro-azionato, come appena osservato, dipende sostanzialmente studiando le proprietà della funzione di guadagno d'anello $L(s) = \G_1(s)\G_2(s)$, e non dal tipo di connessione/ordine tra i due blocchi nello schema.
	
	\subsection{Analisi delle radici del denominatore}
		Un primo modo per studiare la stabilità del sistema retro-azionato è quello di studiare le radici della funzione di trasferimento risultante, definita dalla funzione $1\mp L(s) = D_L(s) + N_L(s) = D_f(s)$, dove $D_f$ rappresenta il denominatore di $G_\textrm{feedback}$ e $N_L,D_L$ rappresentano numeratore e denominatore della funzione di guadagno di anello $L(s)$.
		
		Considerazioni sulla stabilità possono essere dunque effettuate con i metodi giù descritti in precedenza per l'analisi di sistemi dinamici, tuttavia per sistemi di ordine elevato il calcolo delle radici della funzione $ 1 + L(s) = 0$ diventa complesso (in quanto anch'esso di grado elevato) ed è dunque utilizzare dei metodi alternativi basati sui coefficienti.\\
		Si osserva inoltre che i criteri fino ad ora utilizzati non possono essere applicati se nell'anello sono presenti dei sistemi a ritardo di tempo del tipo $1 + L e^{-\tau s}$ (in quanto il denominatore di $\G_f$ non è più un polinomio).
	
	\subsection{Criterio di Nyquist}
		
		Il \de{criterio di Nyquist} è un metodo grafico che permette, sotto forti ipotesi, di determinare la stabilità (o meno) del sistema dinamico osservando il diagramma polare della funzione di trasferimento (dipendente dalla pulsazione $\omega$).
		
		In particolare le ipotesi da verificare per utilizzare il criterio sono:
		\begin{itemize}
			\item non devono essere presenti \textit{parti nascoste} instabili nel calcolo del guadagno di anello $L(s)$;
			\item il sistema deve retro-azionato negativamente.
		\end{itemize}
		
		Il \de{diagramma di Nyquist} di una funzione di trasferimento è l'unione del diagramma polare della funzione $\G(j\omega)$ con il suo specchio rispetto all'asse reale.
		
		Per determinare la stabilità del sistema è necessario calcolare il numero $P$ di poli del guadagno d'anello $L(s)$ con parte reale positiva, il numero $N$ di \textit{giri} del diagramma di Nyquist attorno al punto $-1$ contati positivamente se fatti in senso antiorario. A questo punto \textbf{condizione necessaria} e \textbf{sufficiente} di asintotica stabilità del sistema retro-azionato è che:
		\begin{itemize}
			\item $N$ sia ben definito;
			\item che $N = P$.
		\end{itemize}
		
		Intuitivamente si \textit{gira} attorno al punto $-1$ per via del fatto che $1 + L(s) = 0 \rightarrow L(s) = -1$.
	
		 
		\begin{esempio}{: studio della stabilità tramite il criterio di Nyquist}
			Si consideri un sistema in retroazione negativa la cui funzione d'anello vale  $L(s) = \frac{10}{s+1}$. In questo caso essendo il sistema semplice, basterebbe studiare la stabilità calcolando le radici del polinomio $1 + L(s) = 0$ (che porta al risultato $s=-11$ la cui conclusione è che il sistema è asintoticamente stabile). La soluzione $s=-11$ rappresenta il polo della funzione d'anello; aumentando tale valore è possibile migliorare la velocità della risposta del sistema.
			
			Volendo tuttavia applicare il criterio di Nyquist è necessario rappresentare il diagramma polare; per fare questo è possibile sfruttare delle \textit{regole} qui descritte:
			\begin{enumerate}
				\item verificare che $\G(s)$ non abbia poli sull'asse immaginario (ossia con parte reale nulla): se ciò si verificasse non è possibile rappresentare i diagrammi di Nyquist;
				\item si tracciano i diagrammi di Bode della funzione $G(j\omega)$ che si vuole analizzare;
				\item il diagramma polare viene dunque creato \textit{per punti} scegliendo delle particolari pulsazioni $\omega$, soprattutto per $\omega\rightarrow 0$ e $\omega\rightarrow \infty$. In particolare per $\omega\rightarrow 0$ si osserva che $|\G(j0)| = 20dB = 10$ e la fase vale $\angle G(j0) = 0^\circ$; per $\omega \rightarrow \infty$ si calcola $|\G(j\infty)| = -\infty dB = 0$ e $\angle \G(j\infty) = -90^\circ$. Intuitivamente il diagramma di Nyquist è una linea che unisce i due punti e deve vivere nel 4 semipiano in quanto sia modulo che fase sono sempre decrescenti.
				\item chiudo il diagramma di Nyquist specchiando il diagramma appena tracciato. Il verso di percorrenza è quello delle pulsazioni crescenti sul piano del diagramma polare; sul ramo specchiato il verso della freccia è tale da \textit{non creare scontri}.
			\end{enumerate}
			
			A questo punto è necessario verificare le ipotesi del teorema di Nyquist: essendo $L(s)$ nota, non sono state effettuate \textit{cancellazioni critiche}; inoltre si considera il sistema come in retroazione negativa. A questo punto è necessario calcolare il numero di poli a parte positiva dell'anello che risulta valere $P=0$; questo valore coincide con il numero di giri $N=0$ del diagramma di Nyquist percorsi in senso antiorario rispetto al punto -1 e dunque il \textbf{sistema} è \textbf{asintoticamente stabile}.
		
		\end{esempio}
		
		\paragraph{Estensione del criterio di Nyquist} Supponendo di voler studiare la stabilità di un sistema a retroazione negativa la cui funzione d'anello è moltiplicata per una costante reale del tipo $k\,L(s)$, con $k\in\mathds R$; supponendo di conoscere il diagramma di Nyquist di $L(s)$, per studiare la stabilità è sufficiente considerare che $k$ tende ad espandere/contrarre il diagramma di $L(s)$ già noto.
		
		Tuttavia questa operazione può risultare complessa, quindi alternativamente è possibile pensare di spostare il punto $-1$ in modo da ottenere un risultato relativamente equivalente; in particolare il punto spostato assume valore $-1/k$.
		\begin{nota}
			Tale espressione si ricava partendo dal fatto che il guadagno d'anello da studiare vale $1 + k\,L(s) = 0$; in particolare le due soluzioni equivalenti
			\[ k\, L(s) = -1 \qquad \leftrightarrow \qquad L(s) = - \frac 1 k\]
			Nel primo caso è necessario riscalare il diagramma di $L(s)$, mentre nel secondo è sufficiente spostare il punto $-1$; ai fini del calcolo della stabilità le due operazioni sono equivalenti.
		\end{nota} 
	
		Con questa estensione è possibile rimuovere l'ipotesi di retroazione negativa nel teorema di Nyquist; in particolare per studiare la stabilità di un sistema in retroazione positiva è sufficiente considerare che il denominatore della funzione di trasferimento vale $1 - L(s)$ che è equivalente ad una retroazione negativa con guadagno d'anello pari a $-L(s)$ (ossia equivale allo studio di un anello in retroazione negativa con $k=-1$ e dunque \textit{contando i giri} intorno al punto 1).
		
		\begin{nota}
			Tracciare i diagrammi di Nyquist per funzioni $L$ con poli puramente immaginari è difficile in quanto sostituendo $s\rightarrow \pm j\omega$ nella funzione, si osserva che il modulo di $L$, per una certa pulsazione, tende a valore infinito.
		\end{nota}
		\begin{esempio}{: diag. con polo immaginario}
			Considerando la funzione d'anello $L(s) = \frac 1 s$, è possibile osservare che il grafico è una retta che esplode a valore infinito.
		\end{esempio}
	
	\subsection{Criterio di Bode}
		A questo punto è possibile introdurre questo terzo metodo per lo studio della stabilità della funzione di anello per ovviare alle situazione in cui il disegno della rappresentazione dei diagrammi di Nyquist sia laboriosa; inoltre nonostante non vi siano impedimenti nell'approccio di Nyquist alla funzione $L(s)$ in presenza di ritardi di tempo, nella pratica l'applicazione del teorema risulta impossibile.
	
		Inoltre serve un criterio che possa essere applicato ad una funzione $L(s)$ con poli a parte reale nulla.
		
		\paragraph{Enunciato} Il \de{criterio di Bode} si basa sul tracciamento dei diagrammi Bode e si applica su sistemi in cui si ha il ritardo di tempo e/o in cui ci sono poli di $L(s)$ puramente immaginari. Le ipotesi del teorema sono:
		\begin{itemize}
			\item non devono essere presenti delle parti nascoste di $L(s)$ \textit{critiche};
			\item che il numero di poli a parte positiva $P$ sia sempre nullo (il sistema di partenza deve dunque essere asintoticamente stabili; non può essere applicato a sistemi instabili ad anello aperto);
			\item il diagramma di Bode del modulo di $L(s)$ attraversa l'asse a $0dB$ una volta sola \textit{dall'alto verso il basso} (per frequenze crescenti);
			\item il sistema deve essere a retroazione negativa.
		\end{itemize}
		A questo punto è possibile definire la \textbf{pulsazione critica} $\omega_c$ la pulsazione per la quale il modulo, in decibel, si annulla (oppure $|L(j\omega_c)| = 1$ non in decibel); per calcolare tale valore è possibile:
		\begin{enumerate}[a)]
			\item usare la definizione; 
			\item osservare la pulsazione $\omega_c$ dal diagramma asintotico di Bode del modulo.
		\end{enumerate}
		A questo punto si determina la \textbf{fase critica} $\phi_c = \angle L(j\omega_c)$ da definizione (in quanto i diagrammi asintotici sono troppo approssimati). Dati $\phi_m$ il \textbf{margine di fase} pari a $180^\circ - |\phi_c|$ e il \textbf{guadagno della funzione d'anello} $\mu_L$, allora condizione necessaria e sufficiente di asintotica stabilità del sistema retro-azionato è che sia margine di fase che guadagno d'anello siano entrambi positivi:
		\[ \textrm{asintotica stabilità} \qquad \Leftrightarrow \qquad \phi_m > 0 \qquad \mu_L> 0\]
		
		\paragraph{Interpretazione del criterio di Bode} Questo criterio deriva, di fatto, da quello di Nyquist. I parametri $\mu_L$ e $\phi_m$ di fatto permettono di capire se il valore di $N$ è nullo o meno; noto infatti che $P=0$ applicando il teorema di Nyquist è possibile dedurre direttamente se il sistema è asintoticamente stabile (per $N=0$, mentre se $N\neq 0$ il sistema è instabile).
		
		Considerando il segno del margine di fase $\phi_m$ positivo, allora ad esso corrisponde $N =0$, mentre se esso è negativo $N\neq 0$. \\ Considerando invece il modulo $\mu_L$, se esso è positivo (considerando $\phi_m>0$) si dimostra che $N=0$ riconducendo l'analisi della fase di un guadagno di anello premoltiplicato per un valore negativo contando dunque i giri intorno al punto 1.
		
		Questo criterio è semplice in quanto basta verificare preliminarmente che $\mu_L$ sia positivo: se risultasse negativo si conclude subito che il sistema non è asintoticamente stabile.
		
		\paragraph{Risposta del criterio a sistemi con ritardo di tempo} Considerando un sistema a ritardo di tempo, la prima ipotesi sulle \textit{cancellazioni critiche} è verificata (in quanto non si può semplificare con un polinomio); anche l'ipotesi sul numero di poli a parte reale positiva è verificata (in quanto il ritardo, di per se, non modifica la stabilità del sistema in anello aperto; il calcolo di $P$ dipende solamente dalla parte razionale di $L(s)$); la terza ipotesi rimane verificata in quanto il ritardo di tempo non inficia sul modulo, in decibel, della funzione $L(s)$; anche l'ipotesi sulla retroazione negativa non può essere smentita in quanto dipende dallo schema di controllo (e non dalla presenza del ritardo).
		
		La presenza del ritardo non influenza il calcolo della pulsazione $\omega_c$; il ritardo introduce uno sfasamento che cambia la fase critica $\phi_c$ che tuttavia può essere valutata, partendo dalla pulsazione $\tilde \phi_c$ critica del sistema non ritardato, tramite l'equazione
		\[ \phi_c = \tilde \phi_c - \tau\, \omega_c  \] 
		
		\begin{esempio}{: stabilità con il criterio di Bode}
			Considerando la funzione di guadagno ad anello $L(s) = \frac{10}{s+1}$, la stabilità può essere determinata con il criterio di Bode. Considerando infatti che tutte le  ipotesi sono verificate, dal diagramma del modulo si ricava la pulsazione critica $\omega_c = 10 rad/s$. A questo punto è necessario calcolare la fase critica calcolando $L(j\omega_c)$:
			\[L(j\omega_c) = \frac{10}{j10 + 1} \qquad \Rightarrow\quad \angle L(j\omega_c) = \angle 10 - \angle \big(j\omega_c + 1\big)=0 - \arctan\left( \frac{10}{1} \right) = -84.3^\circ  \]
			\[  \phi_m = 180^\circ - \big|-84.3^\circ \big| = 95.7^\circ \]
			Osservato che il margine di fase $\phi_m$ è positivo  e il guadagno di anello $\mu_L= 10$ è anch'esso positivo, allora è possibile concludere che il sistema retro-azionato è asintoticamente stabile.
		\end{esempio}
		
		\paragraph{Interpretazione del margine di fase}
		\marginnote{18 maggio} Il valore del \textbf{margine di fase} $\phi_m$, definito in precedenza, è legato al massimo ritardo di tempo ammissibile nel sistema retro-azionato.
		
		Immaginando un sistema di controllo retro-azionato con guadagno d'anello $\tilde L(s)$ e ritardo di ingresso pari a $\tau$ (guadagno $e^{-\tau s}$). L'origine di questo ritardo può derivare dalla realizzazione digitale dei sistemi elettronici (per via dei ritardi di comunicazione e tempi di elaborazione). Il margine di fase dunque permette di stabilire qual'è il massimo valore di ritardo $\tau$ oltre il quale il sistema retro-azionato è instabile (nell'ipotesi in cui il sistema non \textit{in ritardo} sia asintoticamente stabile).
		
		Disegnando il diagramma di Nyquist della funzione di trasferimento e calcolati la pulsazione critica $\omega_c$ e il margine di fase $\phi_m$, l'aggiunta del  margine di fase comporta una rotazione del grafico di un valore proporzionale a $\tau$. In particolare lo sfasamento dovuto al ritardo di fase vale $\Delta \phi = -\tau_{max} \omega_c$ e uguagliandolo al margine di fase $\phi_m$ si ottiene che
		\[ \tau_{max} = \frac{\phi_m}{\omega_c} \]
		
		\paragraph{Margine di guadagno} Si consideri il caso di un circuito in retroazione di cui è noto il guadagno d'anello $\tilde L$ che tuttavia è pre-moltiplicato per un guadagno $k$ costante di valore sconosciuto. L'idea è dunque capire quanto vale il valore $k_{max}$ che rende il sistema retro-azionato instabile (nel caso in cui $\tilde L$ sia in origine asintoticamente stabile), e tale valore prende il nome di \de{margine di guadagno}.
		
		Per essere il sistema stabile, allora il numero di giri del diagramma di Nyquist rispetto al punto -1 deve essere nullo; incrementando $k$ di fatto si ha l'espansione del diagramma di Nyquist rispetto all'origine del piano. A questo punto  nota la distanza $a$  del diagramma di Nyquist dall'origine che si puà determinare come $a = |L(j\omega_\pi)$ dove l'angolo $\omega_\pi$ vale $ \angle L(j\omega_\pi) = -180^\circ$. A quesot punto il margine di guadagno vale
		\[ k_m = \frac 1 a \]