\chapter{Introduzione alla teoria del controllo classico}
	Il \textit{mondo reale} nel quale viviamo è descritto nel dominio del tempo $t$, ossia i modelli che caratterizzano i sistemi sono descritti da equazioni differenziali che legano l'uscita $y(t)$ con l'ingresso $u(t)$: l'approccio analitico per il controllo di sistemi nel dominio del tempo prende il nome di \textbf{teoria del controllo moderno} ed è piuttosto \textit{complessa} da analizzare.
	
	\begin{concetto}
		Per \textit{semplificare} l'analisi in questo corso si utilizzerà dunque la \textbf{teoria del controllo classico} che si basa non più sull'analisi dei segnali nel dominio del tempo, ma trasformando i segnali nel dominio nella \textbf{variabile complessa} (o \textbf{\textit{di Laplace}}) $s$ in cui i sistemi, originariamente descritti da equazioni differenziali, sono descritti da equazioni algebriche \textit{ordinarie}, semplificando dunque la trattazione matematica.
		
		Per \textit{trasportare} i segnali dal dominio del tempo a quello della variabile complessa si utilizza la cosiddetta \textbf{trasformata di Laplace} $\L$ e dunque per riportare i segnali dal secondo dominio a quello del tempo si utilizza l'anti-trasformata di Laplace $\aL$.
	\end{concetto} 

	Il vantaggio dell'utilizzare la teoria del controllo classico è che essa è più intuitiva per via del fatto che le equazioni differenziali \textit{scompaiono} e si trasformano in equazioni algebriche che sono più facilmente risolvibili e rappresentabili; questa metodologia infatti è molto simile all'analisi dei sistemi nel dominio della frequenza.
	
	Un problema tuttavia associato a questo tipo di analisi  è che per \textit{leggere} i risultati nel dominio del tempo (interpretabile dall'uomo) è necessario effettuare una doppia operazione di trasformazione e anti-trasformazione nel dominio di Laplace. Questo rende dunque pesante l'analisi di sistemi MIMO; inoltre la rappresentazione di sistemi non lineari nel dominio della variabile di Laplace è piuttosto complessa.

\section{Trasformata di Laplace}
	\begin{concetto}
		Nella teoria del controllo classico l'analisi viene effettuata nel dominio della \textbf{variabile complessa} $s \in \mathds C$ definita da una parte reale $\sigma$ e una parte immaginaria $\omega$, ossia determinata dalla relazione
		\[ s = \sigma + i \omega \qquad \textrm{con } i = \sqrt{-1} \]  
		A questo punto nota una funzione $f(t)$ nel dominio del tempo è possibile definire la sua rispettiva $F(s)$ nel dominio della variabile di Laplace utilizzando la \textbf{trasformata di Laplace} così definita:
		\begin{equation} \label{teoria:eq:trasformata}
			F(s) = \trasf{f(t)} := \int_0^\infty f(t) e^{-st} \, dt
		\end{equation}		
	\end{concetto}
	\begin{osservazione}
		La trasformata di Laplace di alcune funzioni può non esistere: questo può succedere quando il calcolo dell'integrale non porta alla convergenza dello stesso ad un valore finito.
	\end{osservazione}
	
	Convenzionalmente si rappresentano le funzioni nel dominio del tempo con le lettere minuscole (per esempio $f(t)$ ), mentre le trasformate nel dominio della variabile di Laplace si scrivono con la rispettiva lettera maiuscola:
	\[ \textrm{dom. del tempo: } f(t) \qquad \leftrightarrow \qquad \textrm{dom. variabile di Laplace: }\trasf{f(t)} =F(s)  \]
	
	\figura5 1 {scalino}{funzione a scalino.}{scalino}
	\paragraph{Trasformata della funzione a scalino} Si consideri la \textbf{funzione \textit{a scalino}}, mostrata in figura \ref{scalino}, definita per intervalli dall'espressione
	\begin{equation}
	    \scal(t)= \begin{cases} 0 \qquad & t < 0 \\ 1 \qquad & t \geq 0 \end{cases} 
	\end{equation}
    
    Questa funzione risulterà essere particolarmente utile in quanto ogni segnale che verrà analizzato in questo corso sarà considerato come nullo per tempi minori di 0 in quanto si considera la condizione iniziale del sistema completamente descritta dallo stato del sistema stesso (e non dalla storia pregressa degli ingressi). Inoltre tale funzione sarà importante per analizzare sia le \textit{proprietà di equilibrio} sia le \textit{proprietà transitorie} del sistema.
    
    Applicando la definizione della trasformata di Laplace (eq. \ref{teoria:eq:trasformata}) è dunque possibile determinare esplicitamente la funzione nel dominio della variabile complessa $s$:
    \begin{equation} \label{eq:trasf:scalino}
    \begin{aligned}
        F(s) = \trasf{\scal(t)} & = \int_0^\infty \scal(t) e^{-st} \, dt = \int_0^\infty e^-{st}\, dt = \left. - \frac{e^{-st}}{s} \right|_0^\infty \\ & = \frac{1}{s}
    \end{aligned}
    \end{equation}
    
	\figura5 1 {impulso}{funzione impulso.}{impulso}
	
	\paragraph{Trasformata del segnale impulso} La \textbf{funzione \textit{impulso}} $\imp(t)$, rappresentata in figura \ref{impulso}, è determinata da un segnale con valore infinito in un intervallo infinitesimo dello stesso ordine (ossia tale che l'integrale $\int_{-\infty}^\infty \imp(t)\, dt$ sia unitario) che può essere descritta dalla relazione
	\begin{equation}
	    \imp(t) = \lim_{\varepsilon \rightarrow 0} \imp_\varepsilon(t) \qquad \textrm{con } \imp_\varepsilon(t) = \begin{cases}
	    1 / \varepsilon \qquad & 0 \leq t \leq \varepsilon \\ 0 & \textrm{altrove} 
	    \end{cases}
	\end{equation}
	Anche in questo caso è possibile applicare la definizione della trasformata di Laplace per poter stabilire come la funzione impulso si trasforma nel dominio della della variabile complessa $s$:
	\begin{equation} \label{eq:trasf:impulso}
	\begin{aligned}
	    F(s)=\trasf{\imp(t)} &= \lim_{\varepsilon\rightarrow 0} \int_0^\varepsilon \frac 1 \varepsilon e^{-st} \, dt = \lim_{\varepsilon\rightarrow 0} \left( \left. \frac{e^{-st}}{-s\varepsilon} \right|_0^\varepsilon \right) = \lim_{\varepsilon\rightarrow 0} \frac{e^{-s\varepsilon} - 1}{-s\varepsilon} \\ & = 1
	\end{aligned}
	\end{equation}
	
	Questa funzione viene spesso utilizzata per modellare i fenomeni impulsivi (da qui il suo nome) ed è particolarmente importante in quanto, come si dimostrerà nel seguito, sarà importante per stabilire la risposta (nel dominio della variabile complessa $s$) di un qualsiasi sistema lineare per ogni tipo di ingresso applicato per via del fatto che la sua trasformata $\trasf{\imp(t)}$ è sempre costante e pari a 1 (indipendentemente da $s$).
	
	
	\subsection{Proprietà della trasformata di Laplace}
	    La trasformata di Laplace gode di 3 proprietà fondamentali che la rendono uno strumento molto \textit{potente} per analizzare i sistemi dinamici, in particolare si osserva che:
	    \begin{enumerate}[i)]
	        \item l'\textbf{operatore} $\L$ è di tipo \textbf{lineare}, ossia è possibile verificare che
	        \begin{equation}
	            \L\Big( \alpha\, f(t) + \beta\, g(t)\Big) := \alpha\, \trasf{f(t)} + \beta \trasf{g(t)} \qquad \forall \ \alpha,\beta\in\mathds R 
	        \end{equation}
	        
	        \item e' possibile osservare la relazione per la \textbf{traslazione nel dominio del tempo}, ossia dato un ritardo $\tau$ appartenente a tale dominio è possibile verificare la trasformata di $f(t-\tau)$ come
	        \begin{equation}
	            \L \Big( f(t-\tau)\Big) := \trasf{f(t)} e^{-s\tau} = F(s) e^{-s\tau} 
	        \end{equation}
	        
	        \item dualmente alla proprietà ii) è possibile determinare una proprietà di \textbf{traslazione nel dominio della variabile di Laplace} determinata dalla relazione
	        \begin{equation}
	            \trasf{e^{at} f(t) } := F(s-a) \qquad \forall \ a \in \mathds R
	        \end{equation}
	        
	    \end{enumerate}
	
	\subsection{Trasformazioni notevoli}
	    A partire dalle trasformate notevoli della funzione a scalino (eq. \ref{eq:trasf:scalino}) e della funzione impulso (eq. \ref{eq:trasf:impulso}) è possibile determinare le trasformate di altre funzioni di particolare interesso pratico. Considerando per esempio l'\textbf{esponenziale} per un tempo positivo, definito dalla funzione $e^{at}\scal(t)$, sfruttando la proprietà iii) di traslazione nel dominio della variabile complessa si arriva al risultato che
	    \begin{equation}
	        \trasf{e^{at} \scal(t)} = \frac 1 {s-a}
	    \end{equation}
	    
	    \paragraph{Segnali (co)sinusoidali} Considerando un \textbf{segnale cosinusoidale} per soli tempi positivi di pulsazione $\omega$ definito dall'espressione $f(t) = \cos(\omega t)\scal(t)$, per arrivare alla trasformazione nel dominio della variabile di Laplace si ricorre all'utilizzo della trasformazione di Eulero che permette di riscrivere $f(t)$ tramite una notazione fasoriale con i numeri complessi:
	    \[ f(t) = \cos(\omega t) \scal(t) = \frac{e^{i\omega t}+ e^{-i\omega t}}{2} \scal(t)\]
	    
	    Sfruttando dunque la terza proprietà della trasformata $\L$ è possibile ottenere l'analogo della funzione cosinusoidale nel dominio della variabile complessa, in particolare:
	    \begin{equation}
	   	\begin{aligned}
	   		\trasf{f(t)} & = \frac 1 2 \trasf{e^{i\omega t}\scal(t)} + \frac 1 2 \trasf{e^{-i\omega t} \scal(t)} = \frac 1 2 \frac{1}{s-i\omega} + \frac 1 2 \frac{1}{s+i\omega} \\ & = \frac{s}{s^2+\omega^2}
	   	\end{aligned}
	    \end{equation}
	    Seguendo un procedimento analogo è anche possibile determinare la trasformata di un segnale sinusoidale arrivando al risultato che
	    \begin{equation}
	    	\trasf{\sin(\omega t) \scal(t)} = \frac \omega {s^2 + \omega^2}
	    \end{equation}
	    
	    Combinando i risultati appena ottenuti con un andamento esponenziale che moltiplica l'ampiezza del seno (funzione che spesso è possibile trovare nell'analisi di sistemi dinamici in regime transitorio), utilizzando la terza proprietà della trasformata si arriva a verificare che
	    \[ \trasf{e^{at}\cos(\omega t) \scal (t)} = \frac{s-a}{(s-a)^2+\omega^2} \qquad \trasf{e^{at}\sin(\omega t) \scal (t)} = \frac{\omega}{(s-a)^2+\omega^2} \]
	    
	\subsection{Derivate e integrali}
		Una peculiarità di descrivere i sistemi dinamici nel dominio della variabile di Laplace è che tutte le derivate e integrali nel tempo di una funzione generica $f$ vengono ridotte a delle operazioni algebriche.
		\begin{concetto}
			La \textbf{trasformata della derivata} di una funzione generica $\dot f(t)$ (supponendo di conoscere $\trasf f = F(s)$ ) può essere calcolata immediatamente secondo l'operazione
			\begin{equation}
				\trasf{\dot f(t)} := s \, F(s) + f(0)
			\end{equation}
			dove $f(0)$ sono le condizioni iniziali del sistema. Da questa relazione deriva il fatto che l'operatore $s$ è spesso definito \textbf{\textit{operatore derivata}}.
			\vspace{2mm}
			
			Analogamente all'operazione di derivazione è possibile calcolare l'\textbf{integrale} di una funzione utilizzando l'\textbf{\textit{operatore integrale}} $\frac 1 s$ che verifica la relazione
			\begin{equation} \label{eq:trasf:integrale}
				\trasf{\int_0^t f(t)\, dt} : = \frac 1 s F(s)
			\end{equation}
		\end{concetto}
	
		\paragraph{Trasformata di una rampa} Considerando la funzione \textbf{\textit{rampa}} $\ramp(t)$ definita dalla relazione $t \scal(t)$, si può osservare che la funzione rampa coincide con l'integrazione della funzione scalino: sfruttando la proprietà di integrazione (eq. \ref{eq:trasf:integrale}) allora si calcola la trasformata dalla rampa come
		\begin{equation}
			\trasf{\ramp(t)} = \frac 1 s \trasf{\scal(t)} = \frac{1}{s^2}
		\end{equation}
		
		\paragraph{Proprietà di derivazione} Una proprietà conseguente agli operatori di derivazione e integrazione nel dominio del tempo, è il concetto di derivata nel dominio della variabile complessa: in particolare si osserva che essa coincide con la moltiplicazione nel dominio del tempo della funzione originaria secondo l'equazione
		\begin{equation} \label{eq:trasf:propderivazione}
			- \frac d {ds}F(s) = \trasf{t f(t)}
		\end{equation}
		
		
\section{Anti-trasformata di Laplace}
	L'\textbf{anti-trasformata di Laplace}, indicata generalmente come $\aL$, rappresenta l'\textit{operazione inversa} alla trasformazione, ossia è un'operazione che permette di ricavare la funzione $f(t)$ nel dominio del tempo nota la sua corrispettiva $F(s)$ nel dominio della variabile complessa $s$.
	
	Seguirà ora prima l'introduzione di due teoremi principali che permetteranno di stabilire la risposta al tempo iniziale ($t=0$) e asintotica ($t\rightarrow \infty$) di un sistema dinamico, per arrivare successivamente a scrivere la definizione formale dell'anti-trasformata stessa.
	
	\subsection{Teorema del valore iniziale e finale}
		
		\begin{concetto}
			Una \textbf{trasformata} $F(s)$ ottenuta dall'operazione $\trasf{f(t)}$ è detta \textbf{razionale} se e solo se la stessa può essere espressa come un rapporto tra due polinomi nella variabile complessa $s$:
			\[ \textrm{trasformata razionale} \qquad \Leftrightarrow \qquad F(s) := \frac{N(s)}{D(s)} \ \textrm{ con $N(s), D(s)$ polinomi}  \]			
			Di questo tipo di trasformata si denotano \textbf{poli} le radici del polinomio del denominatore $D(s)$, mentre sono detti \textbf{zeri} le radici del polinomio a numeratore $N(s)$.
		\end{concetto}
		\begin{nota}
			A questo livello l'idea di \textit{trasformata razionale} può sembrare \textit{riduttiva} in quanto considera trasformate che possono essere espresse solo come rapporto di polinomi (e non da funzioni generiche). Tuttavia ai fini di questo corso, come è possibile osservare dalle trasformate fin'ora riportate nelle pagine seguenti, i principali segnali di ingresso (e di uscita) sono sempre espressi come polinomi nel dominio della variabile di Laplace. In generale infatti i sistemi lineari determinano sempre trasformate che sono razionali.
		\end{nota}
		\begin{esempio}{: trasformate razionali}
			Un esempio di trasformata $F(s)$ razionale è descritta dalla funzione
			\[ F(s) = \frac{s+1}{s^2 +s-2}\]
			Fattorizzando il denominatore come $(s-1)(s+2)$, allora si evince che i poli di $F(s)$ sono pari a $1$ e $-2$, mentre esiste un unico zero ($N(s)$ è già fattorizzato) pari al valore $-1$.
			
			Al contrario una trasformata $F(s) = e^s \cos(s)$ non è razionale e vedremo che su di essa non si potranno applicare i teoremi che verranno enunciati.
		\end{esempio}
		
		\begin{teorema}{teorema del valore iniziale} \label{teor:valiniziale}
			
			\texttt{Ipotesi:} La trasformata $F(s)$ rispetto al quale si può applicare tale teorema deve essere razionale. \vspace{3mm}
			
			\texttt{Enunciato:} \textit{Data una trasformata di Laplace $F(s)$ razionale, allora il valore della rispettiva funzione $f(t)$ valutata al tempo iniziale $t = 0$ può essere calcolato tramite la relazione}
			\begin{equation}
				f(0) := \lim_{s\rightarrow \infty} s\,F(s)
			\end{equation}
		\end{teorema}
	
		\begin{teorema}{teorema del valore finale} \label{teor:valfinale}
		
			\texttt{Ipotesi:} La trasformata $F(s)$ rispetto al quale si può applicare tale teorema deve essere razionale e si deve verificare che le radici del denominatore $D(s)$ della stessa siano tutte a parte reale negativa non nulla. \vspace{3mm}
			
			\texttt{Enunciato:} \textit{Data una trasformata di Laplace $F(s)$ razionale (che rispetta le ipotesi), allora il valore asintotico della funzione $f(t)$ valutata al tempo $t\rightarrow \infty$ può essere calcolato tramite la relazione}
			\begin{equation}
				\lim_{t\rightarrow \infty}f(t) := \lim_{s\rightarrow 0} s\,F(s)
			\end{equation}
		\end{teorema}
	
	\subsection{Definizione}
		\begin{concetto}
			Data una trasformata $F(s)$ da convertire nel dominio del tempo per determinare $f(t)$ si utilizza l'\textbf{anti-trasformata di Laplace} definita dall'espressione
			\begin{equation}
				f(t) = \trasf{F(s)} := \frac{1}{2\pi i} \int_{\sigma-i\omega}^{\sigma+i\omega} F(s) e^{st} \, dt \qquad \textrm{con $t>\geq 0$ e } \sigma \in \mathds R
			\end{equation}
		\end{concetto}
		Applicare la definizione formale della trasformata di Laplace spesso è \textit{costoso} dal punto di vista operativo: per questo in generale si utilizzano dei \textit{metodi alternativi} per anti-trasformare le funzioni.
		
	\subsection{Metodo di Heaviside}
		\begin{concetto}
			Il \textbf{metodo di Heaviside} è una procedura che viene utilizzata per \textit{semplificare} l'anti-trasformazione di funzioni $F(s)$. Tale sistema, per funzionare, richiede che la funzione $F(s)$ sia razionale e con grado del numeratore inferiore rispetto al grado del denominatore (come si vedrà questa seconda ipotesi potrà essere facilmente \textit{aggirata}) e si basa su:
			\begin{enumerate}
				\item scomporre la trasformata $F(s)$ in una combinazione lineare di trasformate di Laplace note (esponenziali/cosinusoidi... \textit{semplici});
				\item sfruttando la linearità dell'anti-trasformata $\aL$, si calcola $f(t)$ come combinazione lineare delle anti-trasformate note.
			\end{enumerate}
		\end{concetto}	
		Questo metodo sembra \textit{più facile a dirsi che a farsi}, tuttavia nel seguito verranno sviluppati, mediante degli esempi, le anti-trasformate di funzioni di trasferimento piuttosto comuni.
		
		\subsubsection{Poli reali distinti}
			Si consideri una trasformata $F(s)$ (già fattorizzata) a poli reali distinti del tipo
			\[F(s) = \frac{s+2}{s(s+6)(s+1)}\]
			Verificato che $F(s)$ è razionale e che il grado del denominatore (pari a 3) coincide con il grado del denominatore (1), allora è possibile applicare il metodo di Heaviside. In particolare si vuole scomporre la trasformata in 3 elementi (associati ad ogni fattore del polinomio a denominatore) che rendono verificata l'uguaglianza
			\[ F(s) = \frac{s+2}{s(s+6)(s+1)} = \frac A s + \frac{B}{s+6} + \frac{C}{s+1}  \]
			Determinando dunque i coefficienti numerici $A,B,C$ permette di ricavare l'anti-trasformata di $F(s)$ in quanto
			\[ f(t) = \anti{F(s)} = A\scal(t) + Be^{-6t}\scal(t) + C e^{-t}\scal(t) \]
			
			\paragraph{Determinazione dei coefficienti} Per determinare i coefficienti $A,B,C$ che caratterizzano l'anti-trasformata è possibile seguire due possibili \textit{strategie}:
			\begin{itemize}
				\item determinare i coefficienti esplicitando la somma delle funzioni elementari, ossia:
				\begin{align*}
					\frac{s+2}{s(s+6)(s+1)} & = \frac A s + \frac{B}{s+6} + \frac{C}{s+1} = \frac{ A(s+6)(s+1) Bs(s+1) + Cs(s+6) }{s(s+6)(s+1)} \\
					& = \frac{s^2(A+B+C) + s(7A+B+6C)+6A}{s(s+6)(s+1)}
				\end{align*}
				A questo punto per eguagliare il membro a destra con quello a sinistra è necessario imporre l'uguaglianza sui polinomi al numeratore: questo significa risolvere il sistema lineare
				\[ \begin{cases}
					A+B+C = 0 \\ 7A+B+6C = 1 \\ 6A = 2
				\end{cases} \qquad \Rightarrow \quad A = \frac 1 3 \ , \ B = - \frac 2 {15} \ , \ C= -\frac 1 5 \]
				
				\item un secondo metodo per determinare più \textit{agilmente} i coefficienti $A,B,C$ si basa sul fatto che polinomi equivalenti, se valutati nello stesso punto, determinano un risultato uguale. A questo punto si definisce il polinomio del numeratore originario $p_a (s) = s+2$, mentre si definisce il secondo polinomio (partendo dal caso precedente) sommando i contributi iniziali e prendendo in considerazione il numeratore, e dunque $p_b (s) = A(s+6)(s+1) Bs(s+1) + Cs(s+6)$. A questo punto valutando tali polinomi in \textit{punti furbi}, coincidenti di fatto con i poli della funzione $F(s)$, è possibile determinare i coefficienti
				\begin{align*}
					p_a(0) &= p_b(0) \qquad &\Rightarrow 2 & = 6A \qquad & \Rightarrow A &= \frac 1 3  \\
					p_a(-1) &= p_b(-1) \qquad &\Rightarrow 1 & = -5C \qquad & \Rightarrow C &= - \frac 1 5  \\
					p_a(-6) &= p_b(-6) \qquad &\Rightarrow -4 & = -30B \qquad & \Rightarrow B &= - \frac 2 {15}  
				\end{align*}
				
			\end{itemize}
			In entrambi i casi si osserva che l'anti-trasformata della funzione di partenza vale
			\[ f(t) = \anti{\frac{s+2}{s(s+6)(s+1)}} = \frac 1 3\scal(t) -\frac 2 {15}e^{-6t}\scal(t) - \frac 1 5 e^{-t}\scal(t) \]
			
		\subsubsection{Poli reali multipli}
			Data una trasformata $F(s)$ razionale con grado del numeratore inferiore di quella del denominatore con poli reali multipli (e dunque coincidenti), la scomposizione di Heaviside risulta essere analoga al caso di poli reali distinti con l'accortezza di considerare, nella scomposizione in fattori semplici, anche i \textit{termini multipli} associati ai poli coincidenti. Considerando per esempio la funzione
			\[ F(s) = \frac{s+2}{s^2(s+1)}\]
			allora per effettuare la scomposizione di Heaviside è necessario determinare i 3 coefficienti numerici $A,B,C$ che rendono verificata l'uguaglianza
			\[  F(s) = \frac A s + \frac{B}{s^2} + \frac{C}{s+1} \qquad \xrightarrow{\aL} \quad f(t) = A\scal(t) + B\,\ramp(t) + Ce^{-t}\scal(t) \]
			\begin{nota}
				La nostra funzione $F(s)$ di fatto presenta due poli reali coincidenti nel punto $s=0$: per questo nella scomposizione è necessario considerare un denominatore associato alla radice singola (denominatore $s$) e successivamente quello alla radice multipla (denominatore $s^2$).
			\end{nota}
			
			Volendo a questo punto determinare i coefficienti numerici di linearità, è sufficiente utilizzare uno dei due metodi mostrati in precedenza per i poli reali distinti; utilizzando il secondo si determina che
			\[ \frac{s+2}{s^2(s+1)} = \frac{As (s+1)+ B(s+1) +Cs^2}{s^2(s+1)} = \frac{s^2(A+C) + s(A+B) + B}{s^2(s+1)} \]
			\[ \Rightarrow \qquad p_a (s) = s^2 (s+1) \qquad p_b(s) = s^2(A+C) + s(A+B) + B \]
			In questo caso esistendo solamente due \textit{punti furbi} (due poli di coordinate $s = 0,-1$) è possibile determinare solo due coefficienti che risultano valere $B=2$ e $C=1$. Il terzo parametro $A$ si può ricavare valutando i polinomi in un altro punto generico (in quanto gli altri coefficienti sono ora noti) che determina  $ A = -1$. L'anti-trasformata complessiva vale dunque
			\[ f(t) = \anti{F(s)} = -\scal(t) + 2\,\ramp(t) + e^{-t}\scal(t) \]
			
		\subsubsection{Poli complessi coniugati}
			Si consideri la funzione $F(s) = \dfrac{1}{s(s^2+2s+2)}	$ della quale si osservano soddisfatte le condizioni di applicabilità per Heaviside. Cercando di calcolare le radici del denominatore di tale funzione si osserva che le stesse presentano una coppia di numeri complessi coniugati, in particolare i poli si trovano nelle posizioni
			\[ s_1 = 0 \qquad s_{2,3} = \frac{-2\pm \sqrt{4-8}}{2} = -1\pm i\]
			A questo punto è possibile pensare di utilizzare l'approccio di Heaviside utilizzando la fattorizzazione con i numeri complessi e cercando di determinare 3 parametri $A,B,C$ tali che
			\[F(s) = \dfrac{1}{s(s^2+2s+2)} = \frac{1}{s(s+1+i)(s+1-i)} = \frac A s + \frac{B}{s+1+i} + \frac{C}{s+1-i}\]
			Il problema di questo tipo di trasformazione è che risulta \textit{difficile} intuire l'anti-trasformata degli ultimi due termini, e dunque tale scomposizione è poco efficace. Per questo (ma tale idea dovrà essere verificata) si può pensare di scomporre la funzione $F(s)$ come
			\[F(s) = \frac A s + \frac{\beta s + \alpha}{s^2+2s+2} = \frac A s + \gamma \frac{s-a}{(s-a)^2+\omega^2} +\delta \frac{\omega}{(s-a)^2+\omega^2} \]
			Se si riuscisse infatti a determinare il valore dei coefficienti $\gamma,a,\omega$ e $\delta$ sarebbe possibile calcolare l'anti-trasformata di $F(s)$ come
			\[f(t) = \anti{F(s)} = A \scal(t) + \gamma e^{at}\cos(\omega t) \scal(t) + \delta e^{at} \sin(\omega t)\scal(t)\]
			
			\paragraph{Determinazione dei coefficienti} Per determinare i coefficienti numerici di linearità dell'anti-trasformata è necessario in primo luogo calcolare il valore dei parametri $A,\alpha,\beta$; effettuando l'equivalenza delle due espressioni
			\[\frac{1}{s(s^2+2s+2)}  = \frac A s + \frac{\beta s + \alpha}{s^2+2s+2} = \frac{s^2(A+\beta) + s(2A+\alpha) + 2A}{s(s^2+2s+2)}\]
			è possibile determinare i valori numerici dei coefficienti che sono pari ad $A=\frac 1 2,\alpha = -1$ e $\beta = - \frac 1 2$. Determinati questi coefficienti è possibile pensare di risolvere la seconda relazione di equivalenza per la quale
			\[ \xrightarrow[\beta = - 1/2]{\alpha = -1} \quad \frac{-\frac 1 2 s - 1}{s^2+2s+2} = \gamma \frac{s-a}{(s-a)^2+\omega^2} +\delta \frac{\omega}{(s-a)^2+\omega^2} = \frac{\gamma s -\gamma a + \delta \omega }{(s-a)^2+\omega^2} \]
			Eguagliando dunque i polinomi a denominatore si determinano sia il coefficiente $a = -1$ che il termine $\omega= 1$. Conoscendo tali valori è possibile proseguire con la determinazione dei coefficienti $\gamma,\delta$ che risultano essere uguali e pari a $-\frac 1 2$. A questo punto l'anti-trasformata complessiva risulta valere
			\[f(t) = \anti{\frac{1}{s(s^2+2s+2)}} = \frac 1 2 \scal(t) -\frac 1 2 e^{-t}\cos(t)\scal(t) - \frac 1 2 e^{-t}\sin(t)\scal(t)  \]
			
		\subsubsection{Poli complessi coniugati multipli}
			Si consideri la trasformata $F(s) = \dfrac{1}{(s^2+2s+2)^2}	$ molto simile a quella appena analizzata. In questo caso, in modo diverso rispetto alla situazione precedente, è possibile osservare come i poli complessi coniugati siano multipli: il sistema di ordine 4 presenta 4 poli di valori
			\[ s_{1,2,3,4} = -1 \pm j \]
			
			Sfruttando i concetti visti per i poli reali multipli, si può pensare di effettuare una scomposizione di $F(s)$ secondo una forma
			\[ F(s) = \frac{\beta s + \alpha}{s^2+2s+2} + \frac{\dots}{(s^2+2s+2)^2}\]
			Di questa scomposizione il primo termine può essere anti-trasformato secondo la metodologia mostrata per i poli complessi coniugati distinti, mentre meno intuitivo è determinare il numeratore del secondo addendo e capire come lo stesso si anti-trasforma.
			
			Un processo \textit{intuitivo} che si può seguire per anti-trasformare tale elemento si basa sul considerare una funzione $h(t)$ definita come $e^t\sin(\omega t) \scal(t)$ che, per $\omega = 1$, determina un'anti-trasformata pari a 
			\[H(s) = \anti{h(t)} = \frac{1}{(s+1)^2+1} = \frac 1 {s^2+2s+2}\]
			
			Considerando ora la funzione $h'(t) =t \, h(t) = t e^t\sin( t)\scal(t)$, la sua trasformata può essere determinata sfruttando la proprietà di derivazione (eq. \ref{eq:trasf:propderivazione}, pag. \pageref{eq:trasf:propderivazione}) per la quale
			\[ H'(s) = \anti{h'(t)} = - \frac{dH(s)}{ds} = - \Big(-1 \big(s^2+2s+2\big)^{-2}(2s+2)\Big) = \frac{2s+ 2}{\big(s^2 +2s+2\big)^2} \]
			
			Si è cosi mostrato, a livello piuttosto intuitivo e di alto livello, come determinare i coefficienti del numeratore per la scomposizione di Heaviside e dunque anti-trasformare $F(s)$ (l'anti-trasformata è infatti pari alla funzione $h'(t)$).
		
		\subsubsection{Grado del denominatore minore del grado del numeratore}
			
			Specifica richiesta di applicabilità del metodo di Heaviside è che la trasformata $F(s)$, oltre ad essere razionale, presenti un numeratore con grado del polinomio inferiore a quella del denominatore. In realtà questa \textit{specifica} può essere aggirata sfruttando la divisione tra i polinomi: considerando infatti una funzione $F(s) = N(s) / D(s)$ il cui grado del numeratore è maggiore (o uguale) di quello del denominatore, è possibile determinare un coefficiente numerico $\alpha$ e un polinomio $\tilde N(s)$ ottenuto dal resto della divisione in modo che si verifichi
			\[F(s) = \frac{N(s)}{D(s)} = \frac{\tilde N(s)}{D(s)} + \alpha\]
			
			A questo punto la trasformata razionale $\tilde N(s)/D(s)$ può essere anti-trasformata tramite le metodologie precedentemente descritte, mentre il coefficiente numerico $\alpha$ rappresenta un fattore moltiplicativo da anteporre alla funzione impulso, ossia l'anti-trasformata del valore unitario.
			
			\begin{esempio}{: trasformata con grado del numeratore pari a quello del denominatore}
				Si consideri la trasformata $F(s)=\dfrac{s+2}{s+3}$ caratterizzata da un numeratore di grado pari al polinomio a denominatore. Formalmente per tale espressione non sarebbe possibile applicare il metodo di Heaviside, tuttavia è possibile determinare due coefficienti $\alpha,\beta \in \mathds R$ tali da rendere verificata la relazione
				\[\frac{s+2}{s+3} = \frac{\beta}{s+3} + \alpha = \frac{\beta + (s+3)\alpha}{s+2}\]
				Imponendo l'uguaglianza dei numeratori e risolvendo dunque il sistema lineare associato si determinano i parametri $\alpha = 1$, $\beta = -1$ della scomposizione rispetto alla quale si può applicare il metodo di Heaviside con il risultato che
				\begin{align*}
					F(s) & = \frac{-1}{s+3} + 1 \\
					\xrightarrow{ \quad \aL \quad } \qquad f(t) & = - 1 e^{-3t} \scal(t) + \imp(t)
				\end{align*}
				
			\end{esempio}
			
			
			
			
			
			
			
			
			
			
			
			
			
			