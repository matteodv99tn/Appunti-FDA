\chapter{Introduzione alla teoria del controllo classico}
	Il \textit{mondo reale} nel quale viviamo è descritto nel dominio del tempo $t$, ossia i modelli che caratterizzano i sistemi sono descritti da equazioni differenziali che legano l'uscita $y(t)$ con l'ingresso $u(t)$: l'approccio analitico per il controllo di sistemi nel dominio del tempo prende il nome di \textbf{teoria del controllo moderno} ed è piuttosto \textit{complessa} da analizzare.
	
	\begin{concetto}
		Per \textit{semplificare} l'analisi in questo corso si utilizzerà dunque la \textbf{teoria del controllo classico} che si basa non più sull'analisi dei segnali nel dominio del tempo, ma trasformando i segnali nel dominio nella \textbf{variabile complessa} (o \textbf{\textit{di Laplace}}) $s$ in cui i sistemi, originariamente descritti da equazioni differenziali, sono descritti da equazioni algebriche \textit{ordinarie}, semplificando dunque la trattazione matematica.
		
		Per \textit{trasportare} i segnali dal dominio del tempo a quello della variabile complessa si utilizza la cosiddetta \textbf{trasformata di Laplace} $\L$ e dunque per riportare i segnali dal secondo dominio a quello del tempo si utilizza l'anti-trasformata di Laplace $\aL$.
	\end{concetto} 

	Il vantaggio dell'utilizzare la teoria del controllo classico è che essa è più intuitiva per via del fatto che le equazioni differenziali \textit{scompaiono} e si trasformano in equazioni algebriche che sono più facilmente risolvibili e rappresentabili; questa metodologia infatti è molto simile all'analisi dei sistemi nel dominio della frequenza.
	
	Un problema tuttavia associato a questo tipo di analisi  è che per \textit{leggere} i risultati nel dominio del tempo (interpretabile dall'uomo) è necessario effettuare una doppia operazione di trasformazione e anti-trasformazione nel dominio di Laplace. Questo rende dunque pesante l'analisi di sistemi MIMO; inoltre la rappresentazione di sistemi non lineari nel dominio della variabile di Laplace è piuttosto complessa.

\section{Trasformata di Laplace}
	\begin{concetto}
		Nella teoria del controllo classico l'analisi viene effettuata nel dominio della \textbf{variabile complessa} $s \in \mathds C$ definita da una parte reale $\sigma$ e una parte immaginaria $\omega$, ossia determinata dalla relazione
		\[ s = \sigma + i \omega \qquad \textrm{con } i = \sqrt{-1} \]  
		A questo punto nota una funzione $f(t)$ nel dominio del tempo è possibile definire la sua rispettiva $F(s)$ nel dominio della variabile di Laplace utilizzando la \textbf{trasformata di Laplace} così definita:
		\begin{equation} \label{teoria:eq:trasformata}
			F(s) = \trasf{f(t)} := \int_0^\infty f(t) e^{-st} \, dt
		\end{equation}		
	\end{concetto}
	\begin{osservazione}
		La trasformata di Laplace di alcune funzioni può non esistere: questo può succedere quando il calcolo dell'integrale non porta alla convergenza dello stesso ad un valore finito.
	\end{osservazione}
	
	Convenzionalmente si rappresentano le funzioni nel dominio del tempo con le lettere minuscole (per esempio $f(t)$ ), mentre le trasformate nel dominio della variabile di Laplace si scrivono con la rispettiva lettera maiuscola:
	\[ \textrm{dom. del tempo: } f(t) \qquad \leftrightarrow \qquad \textrm{dom. variabile di Laplace: }\trasf{f(t)} =F(s)  \]
	
	\figura5 1 {scalino}{funzione a scalino.}{scalino}
	\paragraph{Trasformata della funzione a scalino} Si consideri la \textbf{funzione \textit{a scalino}}, mostrata in figura \ref{scalino}, definita per intervalli dall'espressione
	\begin{equation}
	    \scal(t)= \begin{cases} 0 \qquad & t \leq 0 \\ 1 \qquad & t \geq 0 \end{cases} 
	\end{equation}
    
    Questa funzione risulterà essere particolarmente utile in quanto ogni segnale che verrà analizzato in questo corso sarà considerato come nullo per tempi minori di 0 in quanto si considera la condizione iniziale del sistema completamente descritta dallo stato del sistema stesso (e non dalla storia pregressa degli ingressi). Inoltre tale funzione sarà importante per analizzare sia le \textit{proprietà di equilibrio} sia le \textit{proprietà transitorie} del sistema.
    
    Applicando la definizione della trasformata di Laplace (eq. \ref{teoria:eq:trasformata}) è dunque possibile determinare esplicitamente la funzione nel dominio della variabile complessa $s$:
    \begin{equation} \label{eq:trasf:scalino}
    \begin{aligned}
        F(s) = \trasf{\scal(t)} & = \int_0^\infty \scal(t) e^{-st} \, dt = \int_0^\infty e^-{st}\, dt = \left. - \frac{e^{-st}}{s} \right|_0^\infty \\ & = \frac{1}{s}
    \end{aligned}
    \end{equation}
    
	\figura5 1 {impulso}{funzione impulso.}{impulso}
	
	\paragraph{Trasformata del segnale impulso} La \textbf{funzione \textit{impulso}} $\imp(t)$, rappresentata in figura \ref{impulso}, è determinata da un segnale con valore infinito in un intervallo infinitesimo dello stesso ordine (ossia tale che l'integrale $\int_{-\infty}^\infty \imp(t)\, dt$ sia unitario) che può essere descritta dalla relazione
	\begin{equation}
	    \imp(t) = \lim_{\varepsilon \rightarrow 0} \imp_\varepsilon(t) \qquad \textrm{con } \imp_\varepsilon(t) = \begin{cases}
	    1 / \varepsilon \qquad & 0 \leq t \leq \varepsilon \\ 0 & \textrm{altrove} 
	    \end{cases}
	\end{equation}
	Anche in questo caso è possibile applicare la definizione della trasformata di Laplace per poter stabilire come la funzione impulso si trasforma nel dominio della della variabile complessa $s$:
	\begin{equation} \label{eq:trasf:impulso}
	\begin{aligned}
	    F(s)=\trasf{\imp(t)} &= \lim_{\varepsilon\rightarrow 0} \int_0^\varepsilon \frac 1 \varepsilon e^{-st} \, dt = \lim_{\varepsilon\rightarrow 0} \left( \left. \frac{e^{-st}}{-s\varepsilon} \right|_0^\varepsilon \right) = \lim_{\varepsilon\rightarrow 0} \frac{e^{-s\varepsilon} - 1}{-s\varepsilon} \\ & = 1
	\end{aligned}
	\end{equation}
	
	Questa funzione viene spesso utilizzata per modellare i fenomeni impulsivi (da qui il suo nome) ed è particolarmente importante in quanto, come si dimostrerà nel seguito, sarà importante per stabilire la risposta (nel dominio della variabile complessa $s$) di un qualsiasi sistema lineare per ogni tipo di ingresso applicato per via del fatto che la sua trasformata $\trasf{\imp(t)}$ è sempre costante e pari a 1 (indipendentemente da $s$).
	
	
	\subsection{Proprietà della trasformata di Laplace}
	    La trasformata di Laplace gode di 3 proprietà fondamentali che la rendono uno strumento molto \textit{potente} per analizzare i sistemi dinamici, in particolare si osserva che:
	    \begin{enumerate}[i)]
	        \item l'\textbf{operatore} $\L$ è di tipo \textbf{lineare}, ossia è possibile verificare che
	        \begin{equation}
	            \L\Big( \alpha\, f(t) + \beta\, g(t)\Big) := \alpha\, \trasf{f(t)} + \beta \trasf{g(t)} \qquad \forall \ \alpha,\beta\in\mathds R 
	        \end{equation}
	        
	        \item e' possibile osservare la relazione per la \textbf{traslazione nel dominio del tempo}, ossia dato un ritardo $\tau$ appartenente a tale dominio è possibile verificare la trasformata di $f(t-\tau)$ come
	        \begin{equation}
	            \L \Big( f(t-\tau)\Big) := \trasf{f(t)} e^{-s\tau} = F(s) e^{-s\tau} 
	        \end{equation}
	        
	        \item dualmente alla proprietà ii) è possibile determinare una proprietà di \textbf{traslazione nel dominio della variabile di Laplace} determinata dalla relazione
	        \begin{equation}
	            \trasf{e^{at} f(t) } := F(s-a) \qquad \forall \ a \in \mathds R
	        \end{equation}
	        
	    \end{enumerate}
	
	\subsection{Trasformazioni notevoli}
	    A partire dalle trasformate notevoli della funzione a scalino (eq. \ref{eq:trasf:scalino}) e della funzione impulso (eq. \ref{eq:trasf:impulso}) è possibile determinare le trasformate di altre funzioni di particolare interesso pratico. Considerando per esempio l'\textbf{esponenziale} per un tempo positivo, definito dalla funzione $e^{at}\scal(t)$, sfruttando la proprietà iii) di traslazione nel dominio della variabile complessa si arriva al risultato che
	    \begin{equation}
	        \trasf{e^{at} \scal(t)} = \frac 1 {s-a}
	    \end{equation}
	    
	    \paragraph{Segnali (co)sinusoidali} Considerando un \textbf{segnale cosinusoidale} per soli tempi positivi di pulsazione $\omega$ definito dall'espressione $f(t) = \cos(\omega t)\scal(t)$, per arrivare alla trasformazione nel dominio della variabile di Laplace si ricorre all'utilizzo della trasformazione di Eulero che permette di riscrivere $f(t)$ tramite una notazione fasoriale con i numeri complessi:
	    \[ f(t) = \cos(\omega t) \scal(t) = \frac{e^{i\omega t}+ e^{-i\omega t}}{2} \scal(t)\]
	    
	    Sfruttando dunque la terza proprietà della trasformata $\L$ è possibile ottenere l'analogo della funzione cosinusoidale nel dominio della variabile complessa, in particolare:
	    \begin{equation}
	   	\begin{aligned}
	   		\trasf{f(t)} & = \frac 1 2 \trasf{e^{i\omega t}\scal(t)} + \frac 1 2 \trasf{e^{-i\omega t} \scal(t)} = \frac 1 2 \frac{1}{s-i\omega} + \frac 1 2 \frac{1}{s+i\omega} \\ & = \frac{s}{s^2+\omega^2}
	   	\end{aligned}
	    \end{equation}
	    Seguendo un procedimento analogo è anche possibile determinare la trasformata di un segnale sinusoidale arrivando al risultato che
	    \begin{equation}
	    	\trasf{\sin(\omega t) \scal(t)} = \frac \omega {s^2 + \omega^2}
	    \end{equation}
	    
	    Combinando i risultati appena ottenuti con un andamento esponenziale che moltiplica l'ampiezza del seno (funzione che spesso è possibile trovare nell'analisi di sistemi dinamici in regime transitorio), utilizzando la terza proprietà della trasformata si arriva a verificare che
	    \[ \trasf{e^{at}\cos(\omega t) \scal (t)} = \frac{s-a}{(s-a)^2+\omega^2} \qquad \trasf{e^{at}\sin(\omega t) \scal (t)} = \frac{\omega}{(s-a)^2+\omega^2} \]
	    
	\subsection{Derivate e integrali}
		Una peculiarità di descrivere i sistemi dinamici nel dominio della variabile di Laplace è che tutte le derivate e integrali nel tempo di una funzione generica $f$ vengono ridotte a delle operazioni algebriche.
		\begin{concetto}
			La \textbf{trasformata della derivata} di una funzione generica $\dot f(t)$ (supponendo di conoscere $\trasf f = F(s)$ ) può essere calcolata immediatamente secondo l'operazione
			\begin{equation}
				\trasf{\dot f(t)} := s \, F(s) + f(0)
			\end{equation}
			dove $f(0)$ sono le condizioni iniziali del sistema. Da questa relazione deriva il fatto che l'operatore $s$ è spesso definito \textbf{\textit{operatore derivata}}.
			\vspace{2mm}
			
			Analogamente all'operazione di derivazione è possibile calcolare l'\textbf{integrale} di una funzione utilizzando l'\textbf{\textit{operatore integrale}} $\frac 1 s$ che verifica la relazione
			\begin{equation} \label{eq:trasf:integrale}
				\trasf{\int_0^t f(t)\, dt} : = \frac 1 s F(s)
			\end{equation}
		\end{concetto}
	
		\paragraph{Trasformata di una rampa} Considerando la funzione \textbf{\textit{rampa}} $\ramp(t)$ definita dalla relazione $t \scal(t)$, si può osservare che la funzione rampa coincide con l'integrazione della funzione scalino: sfruttando la proprietà di integrazione (eq. \ref{eq:trasf:integrale}) allora si calcola la trasformata dalla rampa come
		\begin{equation}
			\trasf{\ramp(t)} = \frac 1 s \trasf{\scal(t)} = \frac{1}{s^2}
		\end{equation}
		
		\paragraph{Proprietà di derivazione} Una proprietà conseguente agli operatori di derivazione e integrazione nel dominio del tempo, è il concetto di derivata nel dominio della variabile complessa: in particolare si osserva che essa coincide con la moltiplicazione nel dominio del tempo della funzione originaria secondo l'equazione
		\begin{equation}
			- \frac d {ds}F(s) = \trasf{t f(t)}
		\end{equation}
		
		
\section{Anti-trasformata di Laplace}
	L'\textbf{anti-trasformata di Laplace}, indicata generalmente come $\aL$, rappresenta l'\textit{operazione inversa} alla trasformazione, ossia è un'operazione che permette di ricavare la funzione $f(t)$ nel dominio del tempo nota la sua corrispettiva $F(s)$ nel dominio della variabile complessa $s$.
	
	Seguirà ora prima l'introduzione di due teoremi principali che permetteranno di stabilire la risposta al tempo iniziale ($t=0$) e asintotica ($t\rightarrow \infty$) di un sistema dinamico, per arrivare successivamente a scrivere la definizione formale dell'anti-trasformata stessa.
	
	\subsection{Teorema del valore iniziale e finale}
		
		\begin{concetto}
			Una \textbf{trasformata} $F(s)$ ottenuta dall'operazione $\trasf{f(t)}$ è detta \textbf{razionale} se e solo se la stessa può essere espressa come un rapporto tra due polinomi nella variabile complessa $s$:
			\[ \textrm{trasformata razionale} \qquad \Leftrightarrow \qquad F(s) := \frac{N(s)}{D(s)} \ \textrm{ con $N(s), D(s)$ polinomi}  \]			
		\end{concetto}
		\begin{nota}
			A questo livello l'idea di \textit{trasformata razionale} può sembrare \textit{riduttiva} in quanto considera trasformate che possono essere espresse solo come rapporto di polinomi (e non da funzioni generiche). Tuttavia ai fini di questo corso, come è possibile osservare dalle trasformate fin'ora riportate nelle pagine seguenti, i principali segnali di ingresso (e di uscita) sono sempre espressi come polinomi nel dominio della variabile di Laplace. In generale infatti i sistemi lineari determinano sempre trasformate che sono razionali.
		\end{nota}
		\begin{esempio}
			Un esempio di trasformata $F(s)$ razionale è descritta dalla funzione
			\[ F(s) = \frac{s+1}{s^2 + 2s + 2}\]
			Al contrario una trasformata $F(s) = e^s \cos(s)$ non è razionale e vedremo che su di essa non si potranno applicare i teoremi che verranno enunciati.
		\end{esempio}
		
		\begin{teorema}{teorema del valore iniziale} \label{teor:valiniziale}
			
			\texttt{Ipotesi:} La trasformata $F(s)$ rispetto al quale si può applicare tale teorema deve essere razionale. \vspace{3mm}
			
			\texttt{Enunciato:} \textit{Data una trasformata di Laplace $F(s)$ razionale, allora il valore della rispettiva funzione $f(t)$ valutata al tempo iniziale $t = 0$ può essere calcolato tramite la relazione}
			\begin{equation}
				f(0) := \lim_{s\rightarrow \infty} s\,F(s)
			\end{equation}
		\end{teorema}
	
		\begin{teorema}{teorema del valore finale} \label{teor:valfinale}
		
			\texttt{Ipotesi:} La trasformata $F(s)$ rispetto al quale si può applicare tale teorema deve essere razionale e si deve verificare che le radici del denominatore $D(s)$ della stessa siano tutte a parte reale negativa non nulla. \vspace{3mm}
			
			\texttt{Enunciato:} \textit{Data una trasformata di Laplace $F(s)$ razionale (che rispetta le ipotesi), allora il valore asintotico della funzione $f(t)$ valutata al tempo $t\rightarrow \infty$ può essere calcolato tramite la relazione}
			\begin{equation}
				\lim_{t\rightarrow \infty}f(t) := \lim_{s\rightarrow 0} s\,F(s)
			\end{equation}
		\end{teorema}
				


